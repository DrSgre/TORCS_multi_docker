% !TEX encoding = UTF-8
% !TEX TS-program = pdflatex
% !TEX root = ../tesi.tex

%**************************************************************
% Sommario
%**************************************************************
\cleardoublepage
\phantomsection
\pdfbookmark{Sommario}{Sommario}
\begingroup
\let\clearpage\relax
\let\cleardoublepage\relax
\let\cleardoublepage\relax

\chapter*{Summary}

In the last years, the development of video game software has become increasingly harder, as their structure is getting ever more complex. As such, the role of Game Engines in this field is now crucial, for allowing a more efficient and effective reuse of functionalities which may be too expensive to develop from scratch for every single project. \\
Still, most popular Game Engines present a monolithic structure, which is problematic for state-of-the-art video game projects that are often aimed towards a multiplayer online environment. Resource scalability, in fact,  is not easy with a monolithic structure and multiple software changes may require constant refactoring of the whole architecture. \\
Our work aims to address the shortcomings of monolithic Game Engines (Legacy Game Engines), by researching and implementing a distributed alternative (Distributed Game Engine), where the modules that compose the software are decoupled and hosted on different virtual containers. \\
Considering the required communication of these modules inside a network environment, we also consider and research the impact of network-specific elements on the performance of such system.
\\ \\
With the aim of preventing possible ambiguity, the technical terms used in the present document are clarified and elaborated on in the \textit{appendix A - Glossary}. Furthermore, in order to facilitate the reading of the document, said terms are marked with a subscript \textit{'G'}.
%\vfill
%
\selectlanguage{english}
%\pdfbookmark{Abstract}{Abstract}
%\chapter*{Abstract}
%
%\selectlanguage{italian}

\endgroup			

\vfill

