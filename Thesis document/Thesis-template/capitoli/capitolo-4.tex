% !TEX encoding = UTF-8
% !TEX TS-program = pdflatex
% !TEX root = ../tesi.tex

%**************************************************************
\chapter{Feasibility study}
\label{cap:feasibility-study}
%**************************************************************

\intro{This chapter discusses the scope of the thesis, with the feasibility study conducted on multiple softwares and introducing the main subject of experimentation, which is the \textbf{TORCS game engine}. \\ \\
	(The main purpose of this whole chapter is to provide the reader with a clear understanding of the reasons behind the choice of TORCS as game engine for our tests.)}

\section{Preliminary analysis}
\intro{This section presents the general study conducted on various software options, in order to evaluate their fitness for the purpose of the project}.

\subsection{Godot}
\intro{Description of Godot with pros and cons with respect to the purpose of the project.}

\subsection{Python Minecraft Clone}
\intro{Description of the Python Minecraft Clone with pros and cons with respect to the purpose of the project.}

\subsection{Flightgear}
\intro{Description of the Flightgear with pros and cons with respect to the purpose of the project.}

\subsection{TORCS - The Open Racing Car Simulator}
\intro{Description of the Torcs with pros and cons with respect to the purpose of the project.}

\subsection{Fofix}
\intro{Description of the Fofix with pros and cons with respect to the purpose of the project.}

\subsection{Conclusion}
\intro{Summary and final decision (on TORCS).}

\section{Technical analysis}
\intro{Techinical analysis of TORCS, considering its main components and state management. Will most likely include multiple subsections dedicated to the various aspects of the software.}

\subsection{Architecture of TORCS}
\intro{In-depth description of the TORCS architecture}.

\subsection{Model View Controller in TORCS}
\intro{To provide additional reasoning about the software structure, we might include the discussion about the \textbf{MVC pattern} applied to TORCS.}

\subsection{Simulated Car Racing Championship Competition (SCR)}
\intro{In-depth description of the SCR client-server architecture, with the action-state communication.}


