% !TEX encoding = UTF-8
% !TEX TS-program = pdflatex
% !TEX root = ../tesi.tex

%**************************************************************
\chapter{Feasibility study}
\label{cap:feasibility-study}
%**************************************************************

\intro{This chapter discusses the scope of the thesis, with a feasibility study conducted on multiple softwares and introducing the main subject of experimentation, which is the \textbf{TORCS game engine}.}

\section{Preliminary analysis}
In order to study the possibility to modularize and distribute a piece of videogame-related software, we need to choose the main subject of research. More specifically, we require a piece of software which is:
\begin{itemize}
	\item \textit{medium sized}, in terms of quantity of code present in its codebase. Software that is too simple can, in fact, limit the amount of features that can be turned into distributed modules. Whereas very complex software could require an excessive amount of effort in order to decouple intertwined functionalities from the main architecture;
	\item \textit{composite}, with a clear distinction between the various parts that make up its codebase. This can be particularly helpful when identifying the functionalities to modularize for the distributed alternative;
	\item \textit{well documented}, particularly in relation to the software architecture. A clear and complete documentation can, in fact, make the process of modularization much more efficient, allowing us to rapidly identify the software components and their interactions.
\end{itemize} 
In addition to these three main criteria, we also try to focus on software which is \textit{not excessively demanding} in terms of resources, since the local hardware limits would be more likely influence the experimental results in that case. \\
We can now elaborate more on the individual software we considered in our preliminary analysis, in the following sections.

\subsection{Godot}
Godot is a cross-platform Game Engine for the development of 2D and 3D games \cite{site:godot-doc}. It provides a comprehensive set of tools, which allow the development of video games that can be easily exported to multiple platforms. \\ \\
This characteristic can prove to be quite interesting for research purposes, as it enables multiple testing scenarios for applications, with different configurations \cite{site:godot-github}. Moreover, this software is described as semi-portable, allowing its executables to be run from any location and never requiring advanced system privileges. \\ \\
This development software supports multiple programming languages, such as:
\begin{itemize}
	\item \textit{GDScript}, which is the recommended Godot original programming language;
	\item \textit{Visual Scripting};
	\item \textit{C\#} and \textit{C++};
	\item \textit{Python}, which is considered to be in an experimental integration phase.
\end{itemize}
These cover most programming languages considered for our research project.
\begin{figure}[h!]
	\centering
	\includegraphics[width=0.9\linewidth]{"immagini/Feasibility study/architecture_diagram"}
	\caption[Representation of the Godot architecture.]{Representation of the Godot architecture.}
	\label{fig:architecture-diagram}
\end{figure}
\\ The engine itself is reasonably well documented \cite{site:godot-doc}, with also a complete graphical representation of its whole architecture \cite{site:godot-architecture}, which we can see in figure \ref{fig:architecture-diagram}. However, this architecture is still quite complex, with multiple layers and many modules very integrated with each other. The core components, despite being quite small-sized, are not necessarily a good fit for modularization purposes, as they are the less flexible part of the structure. \\ \\
Nonetheless, Godot offers flexibility in the form of custom modules \cite{site:godot-modules}, which can be developed in C++ and used to add new functionalities to the original Game Engine. They can be both small-sized or large-sized (e.g. libraries) and used both for video games development and execution. \\ \\
Finally, another feature provided by Godot is the implementation of multi-threading as servers \cite{site:godot-servers}. These structures are daemons which manage data, process it and push the elaborated result to other modules. As such, they can become an interesting component for the purpose of computation offloading in a distributed environment.

\subsection{Python Minecraft Clone}
In the following GitHub repository: \href{https://github.com/obiwac/python-minecraft-clone}{https://github.com/obiwac/python-minecraft-clone} we are provided with the source code of a Python clone of the video game Minecraft. \\
Minecraft is a sandbox game, originally developed in the Java programming language, where the players explore a blocky, procedurally generated 3D world, where they may discover and extract raw minerals, craft tools and items, and build structures \cite{site:minecraft-wiki}. 
\begin{figure}[h!]
	\centering
	\includegraphics[width=0.8\linewidth]{"immagini/Feasibility study/Minecraft"}
	\caption[In-game screenshot of Minecraft.]{In-game screenshot of Minecraft.}
	\label{fig:minecraft}
\end{figure}
\\
Despite not being a Game Engine, this piece of software was deemed to be interesting for our purposes, considering the simplicity of its structure, which would make the identification of the core functionalities relatively easy \cite{site:python-minecraft-github}. However, as discussed in the analysis criteria, the limited amount of features currently implemented could reduce the scope of a possible modularization, especially considering that the project is still in development. \\ \\
In terms of documentation, this piece of software is also quite lacking, with only few comments inside the actual code and no external textual reference. Still, the related video tutorial series \cite{site:python-minecraft-tutorial}, referenced in the repository, is well structured and clearly explains the development of all the software features and components.


\subsection{Flightgear}
Flightgear is an open-source flight simulator written mostly in C++ \cite{site:flightgear-wiki}. This software is quite complex in terms of structure, including a large number of subsystems handling tasks that range from simple features to more complex ones (e.g. multiplayer).
\begin{figure}[h!]
	\centering
	\includegraphics[width=0.8\linewidth]{"immagini/Feasibility study/Flightgear"}
	\caption[In-game screenshot of Flightgear.]{In-game screenshot of Flightgear.}
	\label{fig:flightgear}
\end{figure}
\\ While this complexity aspect is certainly problematic for the modularization of its tasks, the software codebase is managed following programming best practices and with clear distinction between the individual functionalities \cite{site:flightgear-github}. \\
Moreover, the software is extensively documented, with a rich and well maintained Wiki that defines in clear terms what are its main features and tasks \cite{site:flightgear-dev}. \\ \\
Being a "free-exploration" game, there are tasks that are computationally hard to manage and which could benefit from modularization. However, the software itself is quite demanding in terms of computational resources \cite{site:flightgear-performance}. This may cause performance issues that can introduce noise into measurements and evaluations, if the local hardware is not able to satisfy the software requirements.

\subsection{TORCS - The Open Racing Car Simulator}
TORCS is a 3D racing simulator designed to allow AI drivers and humans to compete with each other, with the aim of studying autonomous driving \cite{site:torcs-website}. \\ \\
This software includes both a video game and a Game Engine, with functionalities which are clearly separated inside the codebase \cite{site:torcs-source}. This separation is helpful for a possible modularization of the system, more so if we consider that some of its components are explicitly loaded during execution (plugin), essentially defining the runtime component of the Game Engine (e.g. Rendering, Simulation, Track, Robot). \\ \\
The codebase of this software is medium-sized, with enough functionalities to modularize, but also not excessively complex if the intention is to act on all of its components. Additionally, the system requirements for running simulated races are not high, which prevents performance problems related to insufficient resources and the negative effects that would ensue. \\ \\
The main problem of this software is the lack of maintained documentation, as the provided one is quite outdated and sparse between multiple sources.

\begin{figure}
	\centering
	\includegraphics[width=0.8\linewidth]{"immagini/Feasibility study/TORCS"}
	\caption[In-game screenshot of TORCS.]{In-game screenshot of TORCS.}
	\label{fig:torcs}
\end{figure}


\subsection{FoFiX}
FoFiX is a highly customizable rhythm game supporting many modes of guitar, bass, drum, and vocal gameplay for up to four players. It is the continuation of a long succession of modifications to the original Frets on Fire by Unreal Voodoo \cite{site:fofix-github}. 
\begin{figure}[h!]
	\centering
	\includegraphics[width=0.75\linewidth]{"immagini/Feasibility study/Fofix"}
	\caption[In-game screenshot of FoFiX]{In-game screenshot of FoFiX}
	\label{fig:fofix}
\end{figure}
\\ The codebase itself of this game is medium-sized and written exclusively in the Python programming language. The code contains multiple small tasks, related to specific functionalities that could be benefit from modularization. However, the software was not developed with clear separation between the various functionalities, which are instead quite sparse and disorganized, thus hindering possible modularization operations. \\ \\
There is also little to no real documentation related to the structure of the software, which could be problematic for understanding its features, also considering that the code is quite complex and largely uncommented. \\ \\
Finally, even if the system requirements for this game are quite low, we need to consider that generally managing audio-based software in a network environment is not a trivial task.

\subsection{Conclusion and Final decision}
We now provide a summarization of each software evaluation, in the following table:

\definecolor{lightRowColor}{HTML}{fafafa}
\definecolor{darkRowColor}{HTML}{ffcccb}

\newcommand{\coloredTableHead}{\rowcolor[HTML]{b61827}}
\newcommand{\lightTableRow}{\rowcolor{lightRowColor}}
\newcommand{\darkTableRow}{\rowcolor{darkRowColor}}

\def\arraystretch{1.75}
\rowcolors{2}{lightRowColor}{darkRowColor}
\begin{longtable}{ 
		>{\centering}p{0.20\textwidth} 
		>{}p{0.80\textwidth}}
	
	\caption{Summary of the analysis} \\
	\coloredTableHead
	\textbf{\color{white}Proposal} & 
	\centering\textbf{\color{white}Evaluation}
	\endfirsthead
	
	\rowcolor{white}\caption[]{(continue)}\\
	\coloredTableHead 
	\textbf{\color{white}Proposal} &
	\centering\textbf{\color{white}Evaluation}
	\endhead
	
	Godot & This Game Engine provides a modular structure and interesting features (Godot modules and servers). However, it is unclear if the main components actually fits for the purpose of modularization. \\
	Python Minecraft Clone & The simple structure of the game can be interesting for allowing an easier modularization process. However, in this context, it might be excessive and undermine the purpose of the modularization. \\
	Flightgear & The structure of the code provides many modularization cues, however it exceeds in complexity and resource requirements. \\
	TORCS & The project exposes medium hardware requirements and complexity, while providing a modular codebase structure. The documentation clearly defines the modules actually loaded at runtime, providing an interesting starting point for a further modularization process. \\
	FoFiX & The game is entirely developed in Python, with low requirements and medium size. However the timed audio-based nature of the game, along with the lack of documentation makes it a less interesting option.
\end{longtable}
At the end of this analysis, considering the evaluation criteria and the features of software option, the final choice was on the \textit{TORCS Open Racing Car Simulator}.

\section{Technical analysis}
After TORCS was chosen, as output of the Preliminary analysis, a more in-depth and technical analysis was performed on the actual components of the software. More specifically, we aimed to understand: the role of the various TORCS modules, the general Game Engine logic and how the game state is managed inside the system.

\subsection{Architecture}
The TORCS architecture is composed of three main layers:
\begin{itemize}
	\item \textit{orchestration layer}: which contains the logic used to call the methods defined in the TORCS API and libraries, for general functionalities such as main menu management or game engine initialization;
	\item \textit{API layer}: which contains the actual implementation of the TORCS functionalities, separated in different libraries with specific purposes;
	\item \textit{plugin layer}: which contains the Game Engine modules, related to general functionalities that are loaded and used at runtime (e.g. Rendering, Simulation, Track, Robot).
\end{itemize}
\begin{figure}[h!]
	\centering
	\includegraphics[width=0.8\linewidth]{"immagini/Feasibility study/TORCS architecture"}
	\caption[Representation of the TORCS architecture.]{Representation of the TORCS architecture.}
	\label{fig:torcs-architecture}
\end{figure}
As we can see in the simplified representation of the architecture, on figure \ref{fig:torcs-architecture}, these layers are made of multiple components, which we will now elaborate on in the following sections.

\subsubsection{Orchestration layer}
As previously mentioned, this layer provides the programming logic for controlling the major program flow using different components, including:
\begin{itemize}
	\item a \textit{Entry Stage}, which is the first phase of TORCS start-up operations. In this process, the command line is analysed and, depending on the desired operation mode, TORCS is run either: in command-line mode, with no graphic and real time, or with the graphical menu;
	\item a \textit{Menu} and its components, which are responsible for offering a visual user interface to ease the setup of TORCS. All the settings selected during its usage are persisted in XML files and are read later by the respective components. The components also include a \textit{Race Menu}, which is a special menu that allows the creation of custom racing sessions;
	\item a \textit{State Engine}, which controls the execution of the race specific configurations, including setup, run and shutdown of the simulation. Changes in the TORCS game state can be triggered by data inspection or function calls returned values, allowing the implementation of functionalities which interact with the simulation itself.
\end{itemize}

\subsubsection{API layer}
This layer defines some interfaces and libraries which are used in multiple parts of the project, such as:
\begin{itemize}
	\item XML parameter file handling;
	\item common functions for robots;
	\item features related to portability;
	\item configurations for game screens;
	\item music player functionalities.
\end{itemize}
These include also tools for AI development and creation of \textit{adaptive agents}.

\subsubsection{Plugin layer}
This layer provides multiple plugins which all have specified interfaces and are loaded based on the configurations during runtime. These include Game Engine core functionalities and modules, such as:
\begin{itemize}
	\item \textit{Rendering}: responsible for rendering the current game situation. The default implementation performs visual and audio 3D rendering, based on OpenGL 1.3 and OpenAL APIs;
	\item \textit{Simulation}: responsible for progressing the situation by a given time step;
	\item \textit{Track}: responsible for loading tracks into the TORCS circuit management structure;
	\item \textit{Robot}: responsible for driving cars in the simulation. TORCS can load multiple robots at the same time to drive multiple cars, with one robot supporting up to 10 cars at once. This module can also provide a "Human Driver" (human), which takes input from the user to control the car.
\end{itemize}

\subsection{Simulation and State Management}\label{torcs-state-management}
In the TORCS simulation there are two defined types of time: \textit{simulation time} and \textit{real time}. \\
The simulation time is the time reference with which the simulation-related computations are performed, and its completely independent from real time. The real time, on the other hand, is the time observed by the user, and it's used to synchronize simulation time only in some specific game modes. \\ \\
The TORCS simulation loop starts in the State Engine with an initial update and, if the race is still normally running, the State Engine loop will return into the same state multiple times, till some condition for a state change is fulfilled (e.g. when the race has ended). \\ \\
The simulation updates can be configured with four different operating modes:
\begin{itemize}
	\item \textit{Interactive Mode}: mostly suitable for interactive simulations, taking into account simulation time and real time. In this context, the simulation time is progressed until it has caught up to the real time and then proceeds with rendering one frame;
	\item \textit{Blind Mode}: used for practice sessions, this mode considers only simulation time and no 3D scenery is rendered. The list in the GUI related the current progress is always updated every 2 seconds of simulation time;
	\item \textit{Frame Capturing Mode}: this mode is used to capture frames at exact points in simulation time (e.g. every 0.03s). The time is not synchronized with real time, so the simulation might be faster or slower than real time, depending on the generated load;
	\item \textit{Console Mode}: this mode has no GUI, it only prints on a terminal some progress information.
\end{itemize}
Regardless of the chosen mode, the simulation time step is executed by a Race Engine component during the game main loop. \\ \\
Like most other Game Engines, TORCS manages the information about the current state of the engine with dedicated data structures. More specifically, TORCS game state is divided in three main components:
\begin{itemize}
	\item \textit{Race state}: which is the state related to the current race that is taking place, whether it is started, running, paused or stopped. This state is managed mainly through user GUI or functionalities (e.g. pause game), which are enabled by the core elements of the \texttt{raceengineclient} TORCS library;
	\item \textit{Car state}: different for each car present in a specific race that is taking place, this state contains the information about the current state of the car (e.g. position, damage, speed). This state is updated during each step in the loop of the \texttt{simulation} module;
	\item \textit{Space state}: this component contains the most relevant information about the general game state (e.g. current race time, number of players), even including a reference to the Car state itself. Like the Car state, it is updated during each step in the loop of the \texttt{simulation} module. 
\end{itemize}
In general, the game state management is heavily centralized, with all the state-related structures and variables stored in the \texttt{raceengineclient} library of TORCS. In order to allow other TORCS components to modify the current state, the architecture makes use of C++ pointers, which are provided as parameter of most TORCS method calls. 
\pagebreak

\subsection{Simulated Car Racing Championship Competition (SCR)}\label{scr}
The work presented in the paper by Daniele Loiacono \textit{el al.} proposes an interesting server-multiclient architecture for Simulated Car Recing Championship software \cite{womak:scr-manual}. The goal of this kind of competition is to design a controller for racing car that compete on many tracks alone and/or against other drivers. \\ \\
The basic TORCS architecture comes as a stand-alone application, where the robots are compiled as separate modules and loaded into main memory only when a race takes place. This architecture comes with three major drawbacks:
\begin{itemize}
	\item the races are not in real-time, since the execution of robots is blocking;
	\item there is no separation between the bots and the simulation engine, hence the bots have full access to all the data structures defining the track and the current status of the race;
	\item the only programming languages available for robots development are C and C++.
\end{itemize}
\begin{figure}[h!]
	\centering
	\includegraphics[width=0.75\linewidth]{"immagini/Feasibility study/SCR-architecture"}
	\caption[The architecture of the SCR software.]{The architecture of the SCR software.}
	\label{fig:scr-architecture}
\end{figure}
This proposal involves an evolution of the original TORCS architecture, with three important extensions:
\begin{itemize}
	\item it structures TORCS as a client-server application, where the robots are run as external processes and connected to the race server using UDP;
	\item it introduces real-time races, with each game tick corresponding to about 20 ms of simulated time. In this context, the server sends the current sensory input to each bot and waits for 10 ms (real-time) to receive an action from the robot. Then, if no action arrives, the simulation continues with the last performed action, otherwise a new game state is computed with the new action provided;
	\item the competition software creates a physical separation between the driver code and the race server, with an abstraction layer, thus improving its resilience to cheating.
\end{itemize}
It also adds \textit{scr-server} to the original architecture, which is a component dedicated to managing the connection between the game and a client robot using UDP.
\subsubsection{Sensors and actuators}\label{sensors-actuators}
The input provided by the server to the driver robots consist of data about the current Car state, Race state and Space state. This information can be used by the drivers to compute the next action to send to the server, following the logic of its Artificial Intelligence (AI). \\
In order to obtain such data, the robots use a number of sensors, which come as part of the client structure and are able to read information from the environment surrounding the robot during the races (this includes the presence of opponent cars). \\ \\
Moreover, the robots are able to control the car in the game through a rather typical set of actuators, which includes:
\begin{itemize}
	\item the steering wheel;
	\item the gas pedal;
	\item the brake pedal;
	\item the gearbox.
\end{itemize}
Additionally, a \textit{meta-action} is available to request a race restart to the server.