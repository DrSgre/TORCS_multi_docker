% !TEX encoding = UTF-8
% !TEX TS-program = pdflatex
% !TEX root = ../tesi.tex

%**************************************************************
\chapter{Experiments results \& discussion}
\label{cap:results-discussion}
\intro{In this chapter provides the results of the previously described experiments, with clear representation of data and highlighting the most relevant elements. These results are discussed then in order to provide a correlation with respect to the \textbf{theoretical expectations} and the \textbf{research questions}.}\\

\section{Qualitative experiments}
\intro{The following sections present qualitative experiments, performed in order to obtain a general idea about the effects on the system of specific development decisions. These experiments can provide general numerical data, which is generally \textbf{not} supported by multiple misurations or graphical representations.}

\subsection{ETCD for SCR state-action communication}
\intro{This section presents and discusses the results obtained by the introduction of ETCD for SCR state-action communication, as a replacement for the socket-based original alternative.}

\subsection{Multi-node ETCD cluster}
\intro{This section presents and discusses the results obtained by the introduction of an ETCD cluster of 3 members, verifying the effect this had on the system performance.}

\section{Quantitative experiments}
\intro{The following sections present quantitative experiments, performed in order to obtain precise and numerical data about specific phenomena, effects on the system performance of specific configurations or correlation between system components. The data provided by these experiments is supported by multiple misurations and graphical representations. As such, the structure of the sections may vary depending on the specific experiment.}

\subsection{X11 forwarding performance assessment}
\intro{This section presents and discusses the results obtained by test conducted in order to verify whether X11 can be considered a bottleneck and have significant impact on the performance of video streaming between two different PCs.}

\subsection{Game image streaming solutions}
\intro{This section presents and discusses the results obtained by the experiment performed in order to measure the video streaming performance of the system with different configurations.}

\subsection{Network traffic analysis}
\intro{This section presents and discusses the results obtained by the experiment performed in order to measure the amount of data processed in I/O from/towards the network or the disk, alongside measuring the round-trip-time between SCR client and server in the same architecture.}

\subsection{Network latency impact assessment}
\intro{This section presents and discusses the results obtained by the experiment performed in order to measure the performance of TORCS and screenpipe displays, in situations with high or moderate network latency in the ETCD container.}

\subsection{Distribution of dynamic game state data}
\intro{This section presents and discusses the results obtained by the experiment performed in order to measure the system performance when introducing the storage of an increasing number of game state data fields into ETCD, alongside an increasing amount of network latency in the ETCD container.}

\subsection{Distribution of static game state data}
\intro{This section presents and discusses the results obtained by the experiment performed in order to measure the system performance when introducing the storage of different static game state data fields into ETCD.}

\subsection{ETCD in-memory persistence}
\intro{This section presents and discusses the results obtained by the experiment performed in order to measure the system performance when storing 3 game state data fields in a situation in which ETCD is writing to memory, instead of the Disk.}

\subsection{Graphics and physics engine correlation}
\intro{This section presents and discusses the results obtained by the experiment performed in order to verify and quantify the correlation between the graphic and physics engine of TORCS, by introducing an increasing delay into the simulation module and verifying the impact on the graphics framerate. Moreover, we also compute the theoretical framerate bounds generated by the delays and compare them with the actual measured framerate.}

\subsection{Graphics and game engine framerate correlation}
\intro{This section presents and discusses the results obtained by the experiment performed in order to measure the net operational time and the correlation between the GE framerate and the graphics framerate, in a situation in an increasing delay introduced into the simulation module.}