% !TEX encoding = UTF-8
% !TEX TS-program = pdflatex
% !TEX root = ../tesi.tex
%**************************************************************
\chapter{Experimental methodology}
\label{cap:experimental-methodology}
%**************************************************************

\intro{This chapter discusses the single experiments conducted during the project, providing their \textbf{purpose} and \textbf{expected theoretical outcome. Moreover, this chapter introduces the research methods and design that were used to conduct the studies, both for the \textbf{qualitative} and the \textbf{quantitative} experiments.}}

\section{Qualitative experiments}
\intro{The following sections present qualitative experiments, performed in order to obtain a general idea about the effects on the system of specific development decisions. These experiments can provide general numerical data, which is generally \textbf{not} supported by multiple misurations or graphical representations.}

\subsection{ETCD for SCR state-action communication}
\intro{This section presents the introduction of ETCD for SCR state-action communication, as a replacement for the socket-based original alternative.}
\subsubsection{Purpose \& Expected outcome}
\subsubsection{Methods \& Design}

\subsection{Multi-node ETCD cluster}
\intro{This section presents the introduction of an ETCD cluster of 3 members, verifying the effect this had on the system performance.}
\subsubsection{Purpose \& Expected outcome}
\subsubsection{Methods \& Design}


\section{Quantitative experiments}
\intro{The following sections present quantitative experiments, performed in order to obtain precise and numerical data about specific phenomena, effects on the system performance of specific configurations or correlation between system components. The data provided by these experiments is supported by multiple misurations and graphical representations.}

\subsection{X11 forwarding performance assessment}
\intro{This section presents the test conducted in order to verify whether X11 can be considered a bottleneck and have significant impact on the performance of video streaming between two different PCs.}
\subsubsection{Purpose \& Expected outcome}
\subsubsection{Methods \& Design}

\subsection{Game image streaming solutions}
\intro{This section presents the experiment performed in order to measure the video streaming performance of the system with different configurations.}
\subsubsection{Purpose \& Expected outcome}
\subsubsection{Methods \& Design}

\subsection{Network traffic analysis}
\intro{This section presents the experiment performed in order to measure the amount of data processed in I/O from/towards the network or the disk, alongside measuring the round-trip-time between SCR client and server in the same architecture.}
\subsubsection{Purpose \& Expected outcome}
\subsubsection{Methods \& Design}

\subsection{Network latency impact assessment}
\intro{This section presents the experiment performed in order to measure the performance of TORCS and screenpipe displays, in situations with high or moderate network latency in the ETCD container.}
\subsubsection{Purpose \& Expected outcome}
\subsubsection{Methods \& Design}

\subsection{Distribution of dynamic game state data}
\intro{This section presents the experiment performed in order to measure the system performance when introducing the storage of an increasing number of game state data fields into ETCD, alongside an increasing amount of network latency in the ETCD container.}
\subsubsection{Purpose \& Expected outcome}
\subsubsection{Methods \& Design}

\subsection{Distribution of static game state data}
\intro{This section presents the experiment performed in order to measure the system performance when introducing the storage of different static game state data fields into ETCD.}
	\subsubsection{Purpose \& Expected outcome}
	\subsubsection{Methods \& Design}

\subsection{ETCD in-memory persistence}
\intro{This section presents the experiment performed in order to measure the system performance when storing 3 game state data fields in a situation in which ETCD is writing to memory, instead of the Disk.}
	\subsubsection{Purpose \& Expected outcome}
	\subsubsection{Methods \& Design}


\subsection{Graphics and physics engine correlation}
\intro{This section presents the experiment performed in order to verify and quantify the correlation between the graphic and physics engine of TORCS, by introducing an increasing delay into the simulation module and verifying the impact on the graphics framerate. Moreover, we also compute the theoretical framerate bounds generated by the delays and compare them with the actual measured framerate.}
\subsubsection{Purpose \& Expected outcome}
\subsubsection{Methods \& Design}

\subsection{Graphics and game engine framerate correlation}
\intro{This section presents the experiment performed in order to measure the net operational time and the correlation between the GE framerate and the graphics framerate, in a situation in an increasing delay introduced into the simulation module.}
\subsubsection{Purpose \& Expected outcome}
\subsubsection{Methods \& Design}