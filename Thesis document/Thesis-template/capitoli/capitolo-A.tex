% !TEX encoding = UTF-8
% !TEX TS-program = pdflatex
% !TEX root = ../tesi.tex

%**************************************************************
\chapter{Glossary}
%**************************************************************
\textbf{AI}: Artificial intelligence (AI), refers to the ability of a computer or machine to perform tasks that would normally require human intelligence, such as learning, problem-solving, decision-making, and language understanding. 
\\\\
\textbf{AoI}: in the context of game state storage, an Area of Interest (AoI) is a region in the game world that is being actively used or modified by players.
\\\\
\textbf{API}: an application programming interface (API) is a set of rules, protocols, and tools for building software and applications. It specifies how software components should interact and allows for communication between different systems.
\\\\
\textbf{CLI}: Command-line interface (CLI) is a type of interface that allows users to interact with a computer, device or application by entering commands through a command prompt.
\\\\
\textbf{Cloud}: Cloud (computing) refers to the delivery of computing resources, such as servers, storage, and software, over the Internet. It allows users to access and use these resources on demand, without the need to build and maintain their own infrastructure.
\\\\
\textbf{DB}: referencing Database. A database is a collection of data that is organized in a structured manner and can be accessed electronically. In this document we often refer to Distributed DBs, which are databases spread across multiple machines, often in different locations.
\\\\
\textbf{edge-cloud}: edge-cloud is a term used to describe a distributed cloud computing architecture that brings cloud computing resources and services closer to the edge of the network, closer to the devices and users that need them.
\\\\
\textbf{FPS}: First-person shooter (FPS) is a genre of video games that is characterized by fast-paced action and the use of weapons, typically viewed from a first-person perspective. 
\\\\
\textbf{GE}: referencing Game Engine. A game engine is a software development environment designed for creating video games. It provides a set of tools and frameworks for developers to build games more efficiently by abstracting away lower-level hardware and graphics processing. Game engines often include a physics engine, collision detection, scripting, animation, and other features that are common to most games.
\\\\
\textbf{GUI}: Graphical User Interface (GUI) is a type of user interface that allows users to interact with a computer or device using visual elements such as windows, icons, and menus.
\\\\
\textbf{HTTP}: HTTP (Hypertext Transfer Protocol) is a standard application layer protocol for transmitting hypermedia documents, such as HTML. It is used for communication between clients and servers on the World Wide Web. 
\\\\
\textbf{OS}: an Operating System (OS) is a software program that manages the hardware and software resources of a computer. It provides a platform for running other applications and controls the way in which a user interacts with the computer.
\\\\
\textbf{peer-to-peer}: also known as P2P, it refers to a type of network architecture in which each node or "peer" in the network can act as both a client and a server. In a P2P network, there is no central server that controls the network; instead, each node is able to communicate directly with other nodes and share resources such as data, processing power, or bandwidth.
\\\\
\textbf{PaaS}: Platform as a Service (PaaS) is a type of Cloud computing service that provides a platform for building, deploying, and managing applications over the Internet.
\\\\
\textbf{QoE}: Quality of Experience (QoE) refers to a measure of how well a user is able to use a product or service. In the context of technology, QoE is often used to describe the overall satisfaction and enjoyment that a user experiences when interacting with a device or system.
\\\\
\textbf{RR}: Round robin is a scheduling algorithm that is used to allocate resources fairly among a group of requests or processes. In a round robin system, each request or process is given a turn in a fixed order, and the order is repeated until all requests have been serviced.
\\\\
\textbf{RTT}: Round-Trip Time (RTT) is a measure of the time it takes for a packet of data to be sent from a source to a destination and for a response to be received back at the source.  
\\\\
\textbf{soft-time}: in the context of simulations, "soft time" refers to a type of simulation time that is not tied to real-time clock and can be accelerated or slowed down as needed.
\\\\
\textbf{SSH}: SSH (Secure Shell) is a network protocol that allows secure remote login and other secure network services over an unsecured network.
\\\\
\textbf{SUT}: in the context of software testing, the system under test (SUT) is the software or system that is being tested.
\\\\
\textbf{UDP}: User Datagram Protocol (UDP) is a connectionless protocol that is used to transmit data across a network.
\\\\
\textbf{UI}: User Interface (UI) refers to the means by which a user interacts with a computer or device. It includes the visual elements of the interface, such as the layout, appearance, and visual design, as well as the functionality and control elements, such as buttons, menus, and input fields.
\\\\
\textbf{WAL}: Write-Ahead Log (WAL) is a type of log file that is used to record changes to a database. It is called a "write-ahead" log because the log is written to disk before the corresponding changes are made to the database itself. 
\\\\
\textbf{XML}: Extensible Markup Language (XML) is a markup language that is used to encode data in a text format. It is designed to be both human-readable and machine-readable, and it is often used for storing and transmitting data in a structured manner.
\\\\
\textbf{YAML}: human-readable data serialization language that is commonly used for storing and transmitting data. 



